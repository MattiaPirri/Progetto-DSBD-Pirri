\subsection{Build}
\subsubsection{Make}
Nel caso in cui si volesse usare make, basterà eseguire il seguente comando dalla directory contenente il progetto:
\begin{lstlisting}[language=bash]
    $ make build
\end{lstlisting}
verranno create in automatico tutte le immagini custom necessarie per il progetto.
\subsubsection{Manual build}
Per eseguire manualmente il build delle immagini eseguire i quattro comandi sotto elencati dalla directory contenente il progetto:
\begin{lstlisting}[language=bash]
  $ docker build -t etl-data-pipeline ./Etl\ Data\ Pipeline
\end{lstlisting}
\begin{lstlisting}[language=bash]
  $ docker build -t data-retrieval ./Data\ Retrieval
\end{lstlisting}

\begin{lstlisting}[language=bash]
  $ docker build -t data-storage ./Data\ Storage
\end{lstlisting}

\begin{lstlisting}[language=bash]
  $ docker build -t sla-manager ./SLA\ Manager
\end{lstlisting}
